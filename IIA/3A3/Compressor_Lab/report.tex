\documentclass{article}

% Packages
\usepackage[utf8]{inputenc}

\usepackage{amsmath, bm}
\usepackage{graphicx}
\usepackage{amssymb}
\usepackage{float}
\usepackage{caption}
\usepackage{subcaption}

\title{3A3 Compressor Lab Report}
\author{[Louis Pender]}
\date{\today}

\begin{document}

\maketitle

\section{Abstract}
% Provide a brief summary of the lab report, including the objective, methodology, and key findings.

\section{Theory}
% Describe the experimental setup, equipment used, and the steps followed during the experiment.


% Sketch the compressor rotor marking the inlet and exit flow directions and indicate which side
% of the blade is the suction surface and has the lowest average pressure and which is the pressure
% surface with the highest average pressure. Explain how you can determine which side is which.

% Using the mass flow rate and the impeller measurements, determine the radial velocity at the
% inlet to the blade at maximum flow condition, this is the starting value when valves are fully
% open. Non-dimensionalise your value by the blade speed.

% Draw a velocity triangle for this situation and determine the relative velocity at the inlet to the
% rotor blade. Use this velocity, the inlet diameter of the impeller and the kinematic viscosity of
% water to estimate the Reynolds number for the compressor. The magnitude of this Reynolds
% number will be comparable to values that you are familiar with for pipe flow

% Define a suitable non-dimensional cavitation number. Plot a graph of non-dimensional
% cavitation number against non-dimensional flow coefficient.

\section{Results}
% Present the data obtained from the experiment, including tables, graphs, and figures.

% Plot all the five measured pressure rise characteristics on a single graph. Ensure the graph is
% sufficiently large and has a true zero on both axes. The 5 curves are: 2x full speed and 2x half
% speed for the water circuit and 1x full speed for the glycerine circuit.

\begin{figure}[H]
    \centering
    \includegraphics[width=0.8\textwidth]{compressor_non_dims.png}
    \caption{Example image}
    \label{fig:example-image-a}
\end{figure}

\section{Discussion}
% Analyze and interpret the results, discussing their significance and any observations or trends.

% Explain why throttling the water compressor at inlet and exit produce different cavitation
% results. Also explain why the pressure rise is similar when cavitation is not present.

% Do the results agree with what you expect for:
%% Change of pressure rise with flow rate

%% Change of pressure rise with Reynolds number. Reference to figure 3 may be useful -
%% k/d is comparable to the roughness height of the blade surface divided by the channel span

%% Change in cavitation with rotor speed and throttle location? Sketch two velocity
%% triangles, one at a nominal design condition and when the flow coefficient is much lower.


%$ Can you explain the differences? Note that the two compressors do not give exactly the same
%% pressure rise-flow rate characteristics, and so care must be taken in comparing the effect of
%% Reynolds number.


% Suggest a reason why the venturi might not be suitable for measuring the flow rate of the
% glycerine solution.

\section{Conclusion}
% Summarize the main findings of the experiment and draw conclusions based on the results.

\end{document}
