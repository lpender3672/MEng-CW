
\documentclass[8pt]{article}

\usepackage[utf8]{inputenc}

\usepackage{amsmath, bm}
\usepackage{graphicx}
\usepackage{amssymb}
\usepackage{float}
\usepackage{caption}
\usepackage{subcaption}
% set font size to 11pt

% set margin
\usepackage[margin=0.5in]{geometry}

\setlength{\parskip}{\baselineskip}%
\setlength{\parindent}{0pt}%
\setlength{\headsep}{5pt}

\begin{document}

% insert pdf cover page here


\title{Lab report: 3A1 Drag of bluff and streamlined bodies}
\author{lwp26}
\date{November 2023}
\maketitle

\section{Introduction}

The understanding of drag force for various bodies, and flow regimes is crucial to designing efficient transportation in aeronautical and maritime engineering.

\section{Theory}

The flow is taken to be invicid and in the large section before the contraction of the Markham Tunnel, the velocity is negligible, and so the pressure there is the stagnation pressure $p_0$.
\begin{equation}
    p_0 + 0 = p + \frac{1}{2}\rho U^2 \implies U = \sqrt{\frac{2(p_0-p)}{\rho_a}}
    \label{eq1}
\end{equation}
The pressure difference between the large section and working section, $p_0 - p$ is measured by a pressure transducer.
\begin{equation}
    Re = \frac{Ud}{\nu}
\end{equation}
Non dimensionalising the measured drag.
\begin{equation}
    D = C_d \frac{1}{2} \rho U^2 S \implies C_d = \frac{2D}{ \rho U^2 S}
\end{equation}

Sources of uncertainty in $Re$.
\begin{equation}
    u(Re) = u(U) + u(d) + u(\nu)
\end{equation}
The relative uncertainties, $u(D)$ and $u(\nu)$ are negligible compared to $u(U)$ as the diameter was measured to a higher degree of precision and the viscosity change for the uncertainty in measured temperature is also negligible.
\begin{equation}
    u(Re) \approx u(U) = \frac{1}{2}u(p_0-p) + \frac{1}{2}u(\rho_a)
    \label{eq5}
\end{equation}
Density was calculated from ideal gas law.
\begin{equation}
    \rho_a = \frac{P}{RT} \implies u(\rho_a) = u(T) + u(P)
\end{equation}

Sources of uncertainty in $C_d$, taking $S=\pi d^2/4$.
\begin{equation}
    u(C_d) = u(D) + u(\rho_a U^2) + u(S) = u(D) + u(\frac{1}{2}\rho_a U^2) + 2u(d)
\end{equation}
Substituting equation \ref{eq1} and same reasoning as for equation \ref{eq5} that $u(d)$ is negligible.
\begin{equation}
    u(C_d) \approx u(D) + u(p_0-p)
\end{equation}
The measurement of pressure difference, $p_0-p$, was done with a calibrated pressure sensor, and so the source of absolute uncertainty is taken at the 2 decimal point precision of the sensor.
The uncertainty in drag reading is harder to quantify, as it was observed to fluctuate at higher speeds.
The setup involved buckets of oil to dampen the fluctuations in the drag force, but this was not completely effective.
Single point measurements were taken at each speed, and so the standard deviation of the drag reading is not known.
Instead the uncertainty in the drag reading is taken to be at the 1 point decimal precision, 
which is valid at low speeds where the drag reading is stable.

The Temperature was measured with a temperature gauge with a precision of $1^oC$, this may seem poor but on conversion to Kelvin gives a small relative uncertainty.
The pressure was measured in inches of mercury, with a very high precision of $0.001$ inches.
Changes in mercury's density for the small temperature changes are also negligible and so the atmospheric pressure has negligible uncertainty.

\section{Discussion}

1. Discuss the observations and measurements of the forces and flow fields around the three
bodies. Sketch the flow patterns.

Flow patterns sketched in figure \ref{} show the streamlines around the bodies.

Figure \ref{fig:figure1} shows the plot of $C_d$ vs $Re_D$ for the various bodies. 
For the sphere, the critical Reynolds number where the boundary layer becomes turbulent is marked on the graph.
This shows a sudden drop in the drag coefficient, as the turbulent boundary layer remains attached to the sphere for longer, reducing the size of the wake.
This reduction in wake significantly reduces the pressure drag.
It can also be seen that the sphere with trip wire has a lower drag coefficient at smaller reynolds numbers.
This is because the trip wire causes the boundary layer to become turbulent, and so the boundary layer is turbulent for longer, reducing the size of the wake and hence the pressure drag.


3. Explain why there is a sudden decrease in drag near the critical Reynolds number for the
sphere.

4. Estimate the errors in the drag coefficients at the lowest speed for all the bodies and
mark on your graph. Is the error in the drag coefficient constant? You must include any
equations you use to estimate errors in your report.

The error in the drag coefficient is plotted as a transparent band 

5. Explain the difference in the Cd versus Reynolds number curves for the different bodies in
terms of the flow patterns and the different contributions of form and skin-friction drag.

6. Explain what is meant by fineness ratio and why there is an optimum for a “streamlined
body”.


\begin{figure}[H]
    \centering
    \includegraphics[width=0.8\textwidth]{Re_vs_Cd.png}
    \caption{Graph of drag coefficient against Reynolds number for various bodies}
    \label{fig:figure1}
\end{figure}

\begin{figure}[H]
    \centering
    \begin{subfigure}[t]{0.48\textwidth}
        \centering
        \includegraphics[width=1\textwidth]{Images_Videos/early_seperation.jpg}
        \caption{Seperation before shoulder}
        \label{fig:figure2}
    \end{subfigure}
    ~
    \begin{subfigure}[t]{0.48\textwidth}
        \centering
        \includegraphics[width=1\textwidth]{Images_Videos/late_seperation.jpg}
        \caption{Seperation after shoulder at higher Re}
        \label{fig:figure3}
    \end{subfigure}
\end{figure}

\begin{figure}[H]
    \centering
    \begin{subfigure}[t]{0.48\textwidth}
        \centering
        \includegraphics[width=1\textwidth]{Images_Videos/stream_flat_plate_1.jpg}
        \caption{Streamline drawing}
        \label{fig:figure4}
    \end{subfigure}
    ~
    \begin{subfigure}[t]{0.48\textwidth}
        \centering
        \includegraphics[width=1\textwidth]{Images_Videos/Plate_8milibar.jpg}
        \caption{Tuft visualisation}
        \label{fig:figure5}
    \end{subfigure}
    \caption{Streamlines and tuft visualisation for flat plate}
\end{figure}

\begin{figure}[H]
    \centering
    \begin{subfigure}[t]{0.48\textwidth}
        \centering
        \includegraphics[width=1\textwidth]{Images_Videos/stream_low_Re_sphere.jpg}
        \caption{Streamline drawing}
        \label{fig:figure6}
    \end{subfigure}
    ~
    \begin{subfigure}[t]{0.48\textwidth}
        \centering
        \includegraphics[width=1\textwidth]{Images_Videos/sphere_low_Re.jpg}
        \caption{Tuft visualisation}
        \label{fig:figure7}
    \end{subfigure}
    \caption{Streamlines and tuft visualisation for sphere at low Reynolds number}
\end{figure}

\begin{figure}[H]
    \centering
    \begin{subfigure}[t]{0.48\textwidth}
        \centering
        \includegraphics[width=1\textwidth]{Images_Videos/stream_hi_Re_sphere.jpg}
        \caption{Streamline drawing}
        \label{fig:figure8}
    \end{subfigure}
    ~
    \begin{subfigure}[t]{0.48\textwidth}
        \centering
        \includegraphics[width=1\textwidth]{Images_Videos/sphere_high_Re.jpg}
        \caption{Tuft visualisation}
        \label{fig:figure9}
    \end{subfigure}
    \caption{Streamlines and tuft visualisation for sphere at high Reynolds number}
\end{figure}

\begin{figure}[H]
    \centering
    \begin{subfigure}[t]{0.48\textwidth}
        \centering
        \includegraphics[width=1\textwidth]{Images_Videos/stream_streamlined.jpg}
        \caption{Seperation before shoulder}
        \label{fig:figure10}
    \end{subfigure}
    ~
    \begin{subfigure}[t]{0.48\textwidth}
        \centering
        \includegraphics[width=1\textwidth]{Images_Videos/Streamlined_8milibar.JPG}
        \caption{Seperation after shoulder at higher Re}
        \label{fig:figure11}
    \end{subfigure}
    \caption{Streamlines and tuft visualisation for streamlined body}
\end{figure}

\end{document}
