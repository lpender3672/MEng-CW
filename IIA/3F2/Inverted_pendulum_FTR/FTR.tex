\documentclass{article}

\usepackage[utf8]{inputenc}

\usepackage{amsmath, bm}
\usepackage{graphicx}
\usepackage{amssymb}
\usepackage{float}
\usepackage{caption}
\usepackage{subcaption}
\usepackage{hyperref}
\usepackage{tikz}
\usetikzlibrary{calc}
\usetikzlibrary{angles,quotes} % for pic
\usetikzlibrary{patterns,snakes}
\tikzset{>=latex} % for LaTeX arrow head

\usepackage[margin=0.5in]{geometry}

\setlength{\parskip}{\baselineskip}%
\setlength{\parindent}{0pt}%


% PULLEY
\def\r{0.05} % pulley small radius
\tikzset{
  pics/pulley/.style={
    code={
      \draw[pulcol,line width=0.6] (0,0) circle (#1);
      \draw[pulcol,thick] (0,0) circle (\r);
  }},
  pics/mount/.style args={#1:#2}{ % angle, length
    code={
      \draw[mount] (0,0)++(#1-90:0.9*\r) arc (#1-90:#1-270:0.9*\r) --++ (#1:#2) --++ (#1-90:1.8*\r) -- cycle;
  }},
  pics/weight/.style args={#1,#2,#3}{ % bottom width, top width, height
    code={
      \draw[mass] (0,0) -- (#2/2,0) -- (#1/2,-0.7*#3)
        |- (-#1/2,-#3) -- (-#1/2,-0.7*#3) -- (-#2/2,0) -- cycle;
      \path[mass] (0,0) -- (0,-#3) node[pos=0.52] {$m$};
  }},
  pics/pulley/.default=0.4,
}

\tikzstyle{vvec}=[->,vcol,thick,line cap=round]
\tikzstyle{ground}=[preaction={fill,top color=black!10,bottom color=black!5,shading angle=20},
                    fill,pattern=north east lines,draw=none,minimum width=0.3,minimum height=0.6]
\tikzstyle{metal}=[fill,top color=black!40,bottom color=black!20,shading angle=10]
\tikzstyle{mass}=[line width=0.6,black!30!black,fill=black!40!black!10,rounded corners=1,
                  top color=black!40!black!20,bottom color=black!40!black!10,shading angle=20]
\tikzstyle{pulcol}=[draw=black!30!black,%fill=blue!40!black!10
                    top color=black!40!black!20,bottom color=black!40!black!10,shading angle=20]
\tikzstyle{rope}=[black,thick,line cap=round]
\def\rope#1{ \draw[black,line width=1] #1; \draw[rope] #1; }
\tikzstyle{mount}=[blue!20!black,fill,top color=blue!20!black!70,bottom color=blue!20!black!40,shading angle=10] %,line width=1.8,line cap=round
%\tikzstyle{mount}=[color=black!60,line width=1.8,line cap=round]
%\tikzstyle{spring}=[line width=0.8,black!80,snake=coil,segment amplitude=5,segment length=5,line cap=round]
\tikzstyle{spring}=[thick,decorate,decoration={zigzag,pre length=0.3cm,post length=0.3cm,segment length=6}]

\pgfdeclarelayer{back} % to draw on background
\pgfsetlayers{back,main} % set order

\begin{document}

\title{Full Technical Report: 3F2 Inverted Pendulum Lab}
\author{lwp26}
\date{March 2024}
\maketitle 

\begin{abstract}
    \centering

\end{abstract}

\newpage

\section{Introduction}

\subsection{Purpose}

\subsection{Objectives}
% list the objectives of the experiment

\section{Theory}

\subsection{Inverted Pendulum Dynamical System}

\def\h{0.8}  % mass height
\def\w{0.8}  % mass width
\def\R{0.4}  % pulley radius
\begin{figure}[H]
    \centering
    \begin{tikzpicture}
        \def\H{0.6} % wall height
        \def\T{0.3} % wall thickness
        \def\W{2.6} % ground length
        \def\D{0.2} % ground depth
        \def\x{1.4} % mass width
        \def\pA{-\W+1.8*\R} % A pulley x position
        \def\pB{\W+1.8*\R} % B pulley x position
        \def\Rp{0.25} % pendulum radius
        \def\xp{1.0} % pendulum x position
        \def\L{1.8}  % pendulum length
        \draw[ground] (-\W+1.8*\R+\T/2,0) -- (-\W+1.8*\R+\T/2,\H) --++ (-\T,0) --++ (0,-\H-\D) --++
                    (\T+2.0*\W-\T/2,0) -- (\W + 1.8*\R,0) -- cycle;
        %\draw (0,\H) -- (0,0) -- (\W,0) --++ (0,-\D);
        \rope{(\x,\h/2) -- (\pA,\h/2) arc (90:270:\R) --++ (-\pA,0) coordinate (T)}
        \pic at (\pA,\h/2-\R) {pulley};
        \rope{(\x+\w,\h/2) -- (\pB,\h/2) arc (90:-90:\R) --++ (-\pB,0) coordinate (T)}
        \pic at (\pB,\h/2-\R) {pulley};
        \draw[mass] (\x,0) rectangle ++ (\w,\h);
        \node[above] at (\x+\w/2,\h) {$M$};
        \draw[fill, black] (\x+\w/2, \h/2) circle (0.05);
        \draw[black,line width=1.0] (\x+\w/2, \h/2) coordinate (b) -- (\xp,-\L) coordinate (a) node[midway, left] {$l$};
        \draw[mass] (\xp,-\L) circle (\Rp) ++ (0,0) node {$m$};
        \draw[dashed] (b) -- (\x+\w/2,-\L) coordinate (c);
        %% arrows 
        \draw[<-|] (\x+\w/2,\h+0.5) -- (\pA,\h+0.5) node[midway, above] {$x$};
        \pic[draw, <-, "$\theta$", angle eccentricity=1.5, angle radius=1.2cm] {angle=a--b--c};
        \draw[->,thick] (\pA,0)++(180:1.5*\R) arc (180:90:1.5*\R) node[midway, above] {$T$};

    \end{tikzpicture}

    \caption{Inverted pendulum system.}
    \label{fig:inverted_pendulum}
\end{figure}

\begin{figure}[H]
  \centering

  \begin{tikzpicture}
    \def\H{0.6} % wall height
    \def\T{0.3} % wall thickness
    \def\W{0.8} % ground length
    \def\D{0.2} % ground depth
    \def\x{0.8} % mass x position
    \def\pA{1.8*\R} % pulley x position
    \def\Rp{0.25} % pendulum radius
    \def\xp{0.2} % pendulum x position
    \def\L{1.8}  % pendulum length
    %\draw (0,\H) -- (0,0) -- (\W,0) --++ (0,-\D);

    \pic at (\pA,\h/2-\R) {pulley};
    \draw[->,thick] (\pA,0)++(180:1.5*\R) arc (180:90:1.5*\R) node[midway, above] {$T$};
    \draw[->>,thick] (\pA,0)++(135:3.0*\R) arc (135:90:3.0*\R)  --++ (0.4, -0.1) node[midway, above] {$\ddot{x}/a$};
    \draw[->,thick] (\pA,-\R) -- (\pA+1.0,-\R) node[midway, below] {$F_c$};

    \begin{scope}[xshift=5cm]

      \draw[mass] (\x,0) rectangle ++ (\w,\h);
      \draw[fill, black] (\x+\w/2, \h/2) circle (0.05) ++ (0,0) node[left] {$P$};
      \draw[black,line width=1.0] (\x+\w/2, \h/2) coordinate (b) -- (\xp,-\L) coordinate (a) node[midway, left] {$l$};
      \draw[mass] (\xp,-\L) circle (\Rp) ++ (-0.2,0) node[left] {$Q$};

      %% arrows 
      \draw[<-|] (\x+\w/2,\h+0.5) -- (-1,\h+0.5) node[midway, above] {$x$};
      \draw[->,thick] (\x+\w,\h/2) -- (\x+\w+1.0, \h/2) node[midway, above] {$F_c$};
      \draw[->,thick] (\x,\h/2) -- (\x-1.0, \h/2) node[midway, above] {$F_f$};
      \draw[->,thick] (\x+\w/2, 0) -- (\x+\w/2, -1.5) coordinate (c) node[midway, right] {$Mg$};
      \draw[->,thick] (\xp, -\L-\Rp) --++ (0, -0.6) node[midway, right] {$mg$};
      \draw[->,thick] (\x+\w/2, \h) --++ (0, +1.2) node[midway, right] {$R$};


      \pic[draw, <-, "$\theta$", angle eccentricity=1.5, angle radius=1.2cm] {angle=a--b--c};

    \end{scope}

  \end{tikzpicture}
  \caption{Free body diagrams of the inverted pendulum system.}
  \label{fig:inverted_pendulum_fbd}

\end{figure}
The forces $F_c$, $F_f$ and $R$ are the cable tension, friction force and vertical reaction acting on mass $M$.
The frictional force is defined below
\begin{equation}
  F_f = F \text{sgn}(\dot{x}) \quad \text{where} \quad \text{sgn}(\dot{x}) = \begin{cases} 1 & \dot{x} > 0 \\ -1 & \dot{x} < 0 \\ \text{undefined for } \dot{x} = 0 & \end{cases}
\end{equation}
The vectors $\mathbf{r}_P$ and $\mathbf{r}_Q$ are the position vectors of masses $M$ and $m$ respectively.
\begin{equation}
  \mathbf{r}_P = x \mathbf{i} \quad \text{and} \quad \mathbf{r}_Q = (x-l\sin\theta) \mathbf{i} - l\cos\theta \mathbf{j}
\end{equation}
These positions can be differentiated to find the momentum $\mathbf{p}$ of the body.
\begin{equation}
  \mathbf{p} = \left[(M+m)\dot{x}-ml\dot{\theta}\cos\theta \right] \mathbf{i} + ml\dot{\theta}\sin\theta\mathbf{j}
  \label{eq:trolly_momentum}
\end{equation}
The angular momentum about the point P is given by
\begin{align}
  \mathbf{h}_P &= \sum_i m_i(\mathbf{r}_i - \mathbf{r}_p) \times \mathbf{p}_i \\
  &= m (\mathbf{r}_Q - \mathbf{r}_P) \times \dot{\mathbf{r}}_Q \\
  &= \left[ -ml^2\dot{\theta} + ml\dot{x}\cos\theta \right] \mathbf{k}
\end{align}
This can be differentiated and used with the equation below to find one of the equations of motion 
\begin{equation}
  \mathbf{Q}^{(e)} = \dot{\mathbf{h}}_P + \dot{\mathbf{r}}_P \times \mathbf{p}
\end{equation}
Here the only external force that does not pass through P is the weight of the pendulum bob $mg$ which is a distance $l\sin\theta$ from P,
and so the torque acts in the $+\mathbf{k}$ direction. The equation of motion is then
\begin{equation}
  mgl\sin\theta = ml\ddot{\theta} - ml\ddot{x}\cos\theta
  \label{eq:angular_momentum_deriv_eq}
\end{equation}
From differentiating the linear momentum $\mathbf{p}$ and from newtons second law;
\begin{equation}
  \dot{\mathbf{p}} = \left[F_c - F_f \right] \mathbf{i} + \left[ R - (m+M)g \right]\mathbf{j}
\end{equation}
From considering the unrestrained $\mathbf{i}$ direction the equation of motion is shown to be:
\begin{equation}
  (M+m)\ddot{x} - ml\ddot{\theta}\cos\theta + ml\dot{\theta}^2\sin\theta = F_c - F_f
  \label{eq:x_momentum_deriv_eq}
\end{equation}
However, the force in the cable $F_c$ is required to be in terms of the torque $T$ by the motor.
This is found from considering the angular momentum about the motors axis. The angular acceleration of the pulley is $\ddot{\alpha}$ and the radius of the pulley is $a$.
\begin{align}
  Q &= I\ddot{\alpha} \notag \\
  aF_c - T &= -I\frac{\ddot{x}}{a} \notag \\
  \implies F_c &= \frac{T}{a} - \frac{I}{a}\ddot{x}
\end{align}
This can be then substituted into equation \ref{eq:x_momentum_deriv_eq} and rearanged to give the equation of motion in terms of the torque $T$.
This is shown below along with a simplified form of equation \ref{eq:angular_momentum_deriv_eq}.

\begin{align}
  \left( 1 + \frac{M}{m} + \frac{I}{ma^2} \right) \ddot{x} &= \frac{T}{ma} - \frac{F}{m}\text{sgn}(\dot{x}) + l\cos\theta . \ddot{\theta} - l\sin\theta . \dot{\theta}^2 \label{eq:motion_1} \\
  l \ddot{\theta} &= \cos\theta \ddot{x} - g\sin\theta \label{eq:motion_2}
\end{align}

\subsection{State Space Model}

\begin{align}
  \mathbf{\dot{x}} &= \mathbf{Ax} + \mathbf{Bu} \\
  \mathbf{y} &= \mathbf{Cx}
\end{align}

\subsubsection{Crane Model}

For the crane model the equilibrium point is at $\theta = 0$ and $\dot{\theta} = 0$.
This means that equations \ref{eq:motion_1} and \ref{eq:motion_2} can be linearised about this point to give the equations:
\begin{align}
  \left( 1 + \frac{M}{m} + \frac{I}{ma^2} \right) \ddot{x} &= \frac{T}{ma} - \frac{F}{m}\text{sgn}(\dot{x}) + l \ddot{\theta} \label{eq:crane_motion_1} \\
  l \ddot{\theta} &= \ddot{x} - g\theta \label{eq:crane_motion_2}
\end{align}
These can be solved to give the state space model of the system.
\begin{align}
  \ddot{x} &= \frac{mg}{l\left(M+\frac{I}{a^2}\right)} l\theta + u - f \\
  \ddot{l\theta} &= -\frac{g}{l}l\theta + u - f
\end{align}
Where torque input $u$, and friction disturbance $f$, are defined as follows
\begin{equation}
  u = \frac{T}{a\left(M+\frac{I}{a^2}\right)} \quad \text{and} \quad f = \left(\frac{F}{M + \frac{I}{a^2}} \right) \text{sgn} (\dot{x})
\end{equation}
The state space equations are in a form of simple harmonic motion with natural frequencies shown below,
\begin{align}
  \omega_1^2 &= \frac{g}{l} \\
  \omega_0^2 &= \omega_1^2\left(1 + \frac{m}{M+\frac{I}{a^2}} \right)
\end{align}
This gives the final state space model of the system as

\begin{equation}
  \frac{d}{dt} 
  \begin{bmatrix}
     x \\ \dot{x} \\ l\theta \\ \dot{l\theta} \end{bmatrix} = \begin{bmatrix} 
      0 & 1 & 0 & 0 \\ 0 & 0 & \omega_1^2 - \omega_0^2 & 0 \\ 0 & 0 & 0 & 1 \\ 0 & 0 & -\omega_1^2 & 0 \end{bmatrix} \begin{bmatrix} 
        x \\ \dot{x} \\ l\theta \\ \dot{l\theta} \end{bmatrix} + \begin{bmatrix} 
          0 \\ 1 \\ 0 \\ 1 \end{bmatrix} (u - f)
\end{equation}

\section{Method}

\subsection{Apparatus}


\subsection{Procedure}


\subsection{Uncertainty}



\section{Results}


\section{Discussion}



\section{Conclusion}


\newpage
\section{Appendix}


% bibliography
\newpage
\begin{thebibliography}{9}

\end{thebibliography}

\end{document}