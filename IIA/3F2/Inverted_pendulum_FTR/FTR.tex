\documentclass{article}

\usepackage[utf8]{inputenc}

\usepackage{amsmath, bm}
\usepackage{graphicx}
\usepackage{amssymb}
\usepackage{float}
\usepackage{caption}
\usepackage{subcaption}
\usepackage{hyperref}
\usepackage{tikz}
\usetikzlibrary{calc}
\usetikzlibrary{angles,quotes} % for pic
\usetikzlibrary{patterns,snakes}
\tikzset{>=latex} % for LaTeX arrow head

\usepackage[margin=0.5in]{geometry}

\setlength{\parskip}{\baselineskip}%
\setlength{\parindent}{0pt}%


% PULLEY
\def\r{0.05} % pulley small radius
\tikzset{
  pics/pulley/.style={
    code={
      \draw[pulcol,line width=0.6] (0,0) circle (#1);
      \draw[pulcol,thick] (0,0) circle (\r);
  }},
  pics/mount/.style args={#1:#2}{ % angle, length
    code={
      \draw[mount] (0,0)++(#1-90:0.9*\r) arc (#1-90:#1-270:0.9*\r) --++ (#1:#2) --++ (#1-90:1.8*\r) -- cycle;
  }},
  pics/weight/.style args={#1,#2,#3}{ % bottom width, top width, height
    code={
      \draw[mass] (0,0) -- (#2/2,0) -- (#1/2,-0.7*#3)
        |- (-#1/2,-#3) -- (-#1/2,-0.7*#3) -- (-#2/2,0) -- cycle;
      \path[mass] (0,0) -- (0,-#3) node[pos=0.52] {$m$};
  }},
  pics/pulley/.default=0.3,
}

\tikzstyle{vvec}=[->,vcol,thick,line cap=round]
\tikzstyle{ground}=[preaction={fill,top color=black!10,bottom color=black!5,shading angle=20},
                    fill,pattern=north east lines,draw=none,minimum width=0.3,minimum height=0.6]
\tikzstyle{metal}=[fill,top color=black!40,bottom color=black!20,shading angle=10]
\tikzstyle{mass}=[line width=0.6,black!30!black,fill=black!40!black!10,rounded corners=1,
                  top color=black!40!black!20,bottom color=black!40!black!10,shading angle=20]
\tikzstyle{pulcol}=[draw=blue!30!black,%fill=blue!40!black!10
                    top color=blue!40!black!20,bottom color=blue!40!black!10,shading angle=20]
\tikzstyle{rope}=[black,thick,line cap=round]
\def\rope#1{ \draw[black,line width=1.5] #1; \draw[rope] #1; }
\tikzstyle{mount}=[blue!20!black,fill,top color=blue!20!black!70,bottom color=blue!20!black!40,shading angle=10] %,line width=1.8,line cap=round
%\tikzstyle{mount}=[color=black!60,line width=1.8,line cap=round]
%\tikzstyle{spring}=[line width=0.8,black!80,snake=coil,segment amplitude=5,segment length=5,line cap=round]
\tikzstyle{spring}=[thick,decorate,decoration={zigzag,pre length=0.3cm,post length=0.3cm,segment length=6}]

\pgfdeclarelayer{back} % to draw on background
\pgfsetlayers{back,main} % set order

\begin{document}

\title{Full Technical Report: 3F2 Inverted Pendulum Lab}
\author{lwp26}
\date{March 2024}
\maketitle 

\begin{abstract}
    \centering

\end{abstract}

\newpage

\section{Introduction}

\subsection{Purpose}

\subsection{Objectives}
% list the objectives of the experiment

\section{Theory}

\subsection{Inverted Pendulum System}


\begin{figure}[H]
    \centering
    \def\h{0.6}  % mass height
    \def\w{0.8}  % mass width
    \def\R{0.33}  % pulley radius
    \begin{tikzpicture}
        \def\H{0.6} % wall height
        \def\T{0.3} % wall thickness
        \def\W{2.6} % ground length
        \def\D{0.2} % ground depth
        \def\x{1.4} % mass width
        \def\pA{-\W+1.8*\R} % A pulley x position
        \def\pB{\W+1.8*\R} % B pulley x position
        \def\Rp{0.25} % pendulum radius
        \def\xp{1.4} % pendulum x position
        \def\L{1.2}  % pendulum length
        \draw[ground] (-\W+1.8*\R+\T/2,0) -- (-\W+1.8*\R+\T/2,\H) --++ (-\T,0) --++ (0,-\H-\D) --++
                    (\T+2.0*\W-1.8*\R-\T/2,0) -- (\W,0) -- cycle;
        %\draw (0,\H) -- (0,0) -- (\W,0) --++ (0,-\D);
        \rope{(\x,\h/2) -- (\pA,\h/2) arc (90:270:\R) --++ (-\pA,0) coordinate (T)}
        \pic at (\pA,\h/2-\R) {pulley};
        \rope{(\x+\w,\h/2) -- (\pB,\h/2) arc (90:-90:\R) --++ (-\pB,0) coordinate (T)}
        \pic at (\pB,\h/2-\R) {pulley};
        \draw[mass] (\x,0) rectangle++ (\w,\h) node[midway] {$M$};
        \draw (\x+\w/2, \h/2) -- (\xp,-\L);
        \draw[mass] (\xp,-\L) circle (\Rp) ++ (0,0) node {$m$};
    \end{tikzpicture}

    \caption{Inverted pendulum system.}
    \label{fig:inverted_pendulum}
\end{figure}

\subsection{State Space Model}


\subsection{Compressible flow theory}

\subsection{Visualisation Techniques}


\section{Method}

\subsubsection{Apparatus}


\subsubsection{Procedure}


\subsection{Uncertainty}



\section{Results}


\section{Discussion}



\section{Conclusion}


\newpage
\section{Appendix}


% bibliography
\newpage
\begin{thebibliography}{9}

\end{thebibliography}

\end{document}