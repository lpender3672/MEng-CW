
\documentclass{article}

\usepackage[utf8]{inputenc}

\usepackage{amsmath}
\usepackage{graphicx}
\usepackage{amssymb}
\usepackage{float}

\setlength{\parskip}{\baselineskip}%
\setlength{\parindent}{0pt}%

\begin{document}

\title{Boundary Layers}
\author{lwp26 }
\date{October 2022}
\maketitle

\section{Abstract}
In this report a model structure with 3 degrees of freedom will be numerically modelled and analysed to determine optimal design parameters
for three seperate absorbers to reduce the amplitude of the structure's resonant response to harmonic excitations at each mode shape.
This approach can be used on real structures to reduce the amplitude of their resonant response to earthquakes.

\section{Abstract}
In this report a model structure with 3 degrees of freedom will be hamonically excited to determine the resonant frequencies and mode shapes of the structure.
The resonant frequnencies and mode shapes will be used to determine optimal design parameters for three seperate absorbers to reduce the resonant responses at each mode shape.
% The optimal design parameters will be determined by using a genetic algorithm to minimise the resonant response of the structure.
This approach can be used on real structures to reduce the amplitude of their resonant response to earthquakes.

\section{Introduction}

% Aims, Objectives and context

\subsection{Aims}

\begin{itemize}
\item To numerically model a 3 degree of freedom structure with 3 seperate absorbers to reduce the amplitude of the structure's resonant response to harmonic excitations represting earthquakes.
\item To
\end{itemize}

\section{Methodology}

% Summary of theory and information to reproduce

\section{Results}

\section{Discussion}

%interpret results and comment on anomalies

\section{Conclusion}

\end{document}