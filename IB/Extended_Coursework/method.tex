

\documentclass[8pt]{article}

\usepackage[utf8]{inputenc}

\usepackage{amsmath}
\usepackage{graphicx}
\usepackage{amssymb}
\usepackage{float}
% set font size to 11pt


\setlength{\parskip}{\baselineskip}%
\setlength{\parindent}{0pt}%

\begin{document}

\title{Method Sheet}
\author{lwp26}
\date{Feburary 2023}
\maketitle

\newpage

\section{Method and theory}
% Summary of theory and information to reproduce

The structure experimented on in this report has 3 floors stacked on top of each other connected by springs.
Resonant frequencies were determined by performing a sweep of harmonic excitations on the structure and recording the responses at each frequency.
These undamped responses were then saved and used for comparison with the responses of the structure after the addition of tuned mass dampers.

The tune mass damper model used is a simple spring-mass-damper system with a spring stiffness $k$ and damping coefficient $\lambda$.
The position of the damper is also important as it must be ideally placed at an anti node of the resonant mode shape.
For each mass spring damper the stiffness, $k$, must be tuned to match the resonant frequency of the structure.
\begin{equation}
    k = m\omega_n^2
\end{equation}
The spring used is effectively a cantilever which has a stiffness
\begin{equation}
    k = \frac{W}{\delta} = \frac{3EI}{L^3}
\end{equation}
Equating these gives the length of the cantilever as
\begin{equation}
    L = \sqrt[3]{\frac{3EI}{m\omega_n^2}}
\end{equation}
This did not however give optimal results for a variety of reasons, the main one being that the equation for deflection
of a cantilever assumes small deflections and the deflection of the tuned mass damper is significant.
The method used for tuning was to incrementally change the stiffness of the spring by changing its length until the response 
at the resonant frequency was at a minimum. 

A single mass of 50g was added to the end of the cantilever spring and the structures response was tested with the mass at the top and 1cm below the top.
In the 1st mode the response was worse when the mass was positioned 1cm from the top and so two more 50g masses were added until the resonant response decreased.
This was done such that the optimal length lies on the cantilever for fine tuning. It can also be seen in equation (3) that increasing $m$
decreases the optimal length of the cantilever such that the tuned mass dampers resonant frequency matches that of the building.
The other modes did not require this as a single 50g mass was sufficient to 

The damping coefficient was effectively unchanged as this is caused by the viscous friction of the air on the mass.


\end{document}