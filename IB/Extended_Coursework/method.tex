

\documentclass[8pt]{article}

\usepackage[utf8]{inputenc}

\usepackage{amsmath}
\usepackage{graphicx}
\usepackage{amssymb}
\usepackage{float}
% set font size to 11pt


\setlength{\parskip}{\baselineskip}%
\setlength{\parindent}{0pt}%

\begin{document}

\title{Extended Coursework: Method Sheet}
\author{lwp26}
\date{Feburary 2023}
\maketitle


\section{Method and theory}
% Summary of theory and information to reproduce

The structure experimented on in this report has 3 floors stacked on top of each other connected by springs.
Resonant frequencies were determined by performing a sweep of harmonic excitations on the structure and recording the responses at each frequency.
These undamped responses were then saved and used for comparison with the responses of the structure after the addition of tuned mass dampers.

The tune mass damper model used is a simple spring-mass-damper system with a spring stiffness $k$ and damping coefficient $\lambda$.
The position of the damper is also important as it must be ideally placed at an anti node of the resonant mode shape.
The tuned mass damper for mode 1 was on the 3rd floor as this floor had the largest amplitude at the resonant frequency.
The tuned mass dampers for modes 2 and 3 were placed on the 1st and 2nd floor respectively as both modes excited floor 2, only mode 2 excited floor 1.
For each mass damper the cantilever spring stiffness, $k$, must be tuned to match the resonant frequency of the structure.
\begin{equation}
    k = m\omega_n^2 = \frac{W}{\delta} = \frac{3EI}{L^3}
\end{equation}
Where $\delta$ is the deflection of the cantilever spring
\begin{equation}
    L = \sqrt[3]{\frac{3EI}{m\omega_n^2}}
\end{equation}
This did not however give optimal results for a variety of reasons, the main one being that the equation for deflection
of a cantilever assumes small deflections and the deflection of the tuned mass damper is significant.
The method used for tuning was to incrementally change the stiffness of the spring by changing its length until the response 
at the resonant frequency was at a minimum. 

The mass on the cantilever was chosen such that the optimal length lied on the cantilever for fine tuning.
This was done using equation (3) where it can be seen by increasing $m$ it decreases the optimal length of the cantilever such 
that the tuned mass dampers resonant frequency matches that of the building.

The damping coefficient was effectively unchanged as this is caused by the viscous friction of the air on the mass.


\end{document}