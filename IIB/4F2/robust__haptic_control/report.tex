\documentclass{article}

\usepackage[utf8]{inputenc}

\usepackage{amsmath, bm}
\usepackage{graphicx}
\usepackage{amssymb}
\usepackage{float}
\usepackage{caption}
\usepackage{subcaption}
\usepackage{hyperref}
\usepackage{tikz}
\usepackage{pgfgantt}
\usepackage{layout}

\usepackage[margin=1in]{geometry}
\usepackage{listings}
\usepackage{xcolor}
\usepackage{color, colortbl}
\usepackage{textgreek}
\usepackage{mathrsfs}
\usepackage{savetrees}

\usetikzlibrary{calc}
\usetikzlibrary{angles,quotes} % for pic
\usetikzlibrary{patterns,snakes}
\usetikzlibrary{arrows}
\usetikzlibrary{shapes.geometric, arrows}
\tikzset{>=latex} % for LaTeX arrow head

\setlength{\parskip}{\baselineskip}%
\setlength{\parindent}{0pt}%
\linespread{0.9}


\definecolor{codegreen}{rgb}{0,0.6,0}
\definecolor{codegray}{rgb}{0.5,0.5,0.5}
\definecolor{codepurple}{rgb}{0.58,0,0.82}
\definecolor{backcolour}{rgb}{0.95,0.95,0.92}

\lstdefinestyle{mystyle}{
    backgroundcolor=\color{backcolour},   
    commentstyle=\color{codegreen},
    keywordstyle=\color{magenta},
    numberstyle=\tiny\color{codegray},
    stringstyle=\color{codepurple},
    basicstyle=\ttfamily\footnotesize,
    breakatwhitespace=false,         
    breaklines=true,                 
    captionpos=b,                    
    keepspaces=true,                 
    numbers=left,                    
    numbersep=5pt,                  
    showspaces=false,                
    showstringspaces=false,
    showtabs=false,                  
    tabsize=2
}
\tikzset{
    cg/.style={
        draw,
        circle,
        thick,
        minimum size=0.2cm, % Adjust the size of the circle
        inner sep=0,      % Remove extra padding
        append after command={
            \pgfextra{
                \draw[thick] (\tikzlastnode.south) -- (\tikzlastnode.north);
                \draw[thick] (\tikzlastnode.west) -- (\tikzlastnode.east);
            }
        }
    }
}

\lstset{style=mystyle}

\tikzstyle{startstop} = [rectangle, rounded corners, minimum width=3cm, minimum height=1cm,text centered, draw=black, fill=red!30]
\tikzstyle{process} = [rectangle, minimum width=3cm, minimum height=1cm, text centered, draw=black, fill=orange!30]
\tikzstyle{decision} = [diamond, minimum width=3cm, minimum height=1cm, text centered, draw=black, fill=green!30]
\tikzstyle{io} = [trapezium, trapezium left angle=70, trapezium right angle=110, minimum width=3cm, minimum height=1cm, text centered, draw=black, fill=blue!30]
\tikzstyle{arrow} = [thick,->,>=stealth]

\begin{document}
\immediate\write18{py mergebib.py}

\title{4F2 Robust and Nonlinear control}
\author{5735G}
\date{Feburary 2025}
\maketitle 

\begin{center}
    \textbf{Summary} \\
    I dont know what to do for this
\end{center}

%-----------------------------------------------------------------------------------------
\section{Motivation}
%-----------------------------------------------------------------------------------------
pass


\section{Application to problem}

The dynamics of a soft link robot can be described by the following equation of motion:

\begin{equation}
    m\ddot{y} + c_v \dot{y} + c_p y = c_p u
\end{equation}

\begin{table}[h]
    \centering
    \begin{tabular}{c|ll|l}
        Variable & Value & Uncertainty & Description \\
        \hline
        $m$ & 1.0 & $\pm0$ & link mass \\
        $c_v$ & 1.0 & $\pm 0.1$ & mechanical dissipation \\
        $c_p$ & 1.0 & $\pm 0.075$ & represents spring stiffness \\
    \end{tabular}
    \caption{System variables}
    \label{tab:example}
\end{table}

The state space representation of the system in the form $\dot{x} = Ax + Bu$ and $y = Cx + Du$ is given by:

\begin{align}
    \frac{d}{dt} \begin{bmatrix}
        x \\
        \dot{x}
    \end{bmatrix} &= \begin{bmatrix}
        0 & 1 \\
        -\frac{c_p}{m} & -\frac{c_v}{m}
    \end{bmatrix} \begin{bmatrix}
        x \\
        \dot{x}
    \end{bmatrix} + \begin{bmatrix}
        0 \\
        \frac{c_p}{m}
    \end{bmatrix} u \\
    y &= \begin{bmatrix}
        1 & 0
    \end{bmatrix} \begin{bmatrix}
        x \\
        \dot{x}
    \end{bmatrix} + 0u
\end{align}

The characteristic open and closed loop equations are given by $\det(sI - A) = 0$ and $\det(sI - (A-BKC)) = 0$.

\begin{align}
    s^2 + \frac{c_v}{m}s + \frac{c_p}{m} &= 0 \\
    s^2 + \frac{c_v}{m}s + (1+k)\frac{c_p}{m} &= 0
\end{align}
The Routh-Hurwitz criterion for second order characteristic polynomial states that the system is stable if all coefficients are positive.
This shows that the open loop is always stable, but the closed loop system is only stable if $k > -1$.

\begin{figure}[H]
    \centering
    \includegraphics[width=0.6\textwidth]{figures/bode_G.png}
    \caption{}
    \label{fig:bode_G}
\end{figure}

\begin{figure}[H]
    \centering
    \includegraphics[width=0.6\textwidth]{figures/nyquist_G.png}
    \caption{}
    \label{fig:nyquist_G}
\end{figure}

\section{Tracking performance}


The error transfer function of the closed loop is found by evaluating $E(s) = 1/(1+kG(s))$.
For the steady state error to a step response is found by $\frac{1}{s} \cdot E(s)$ and the final value theorem.
\begin{equation}
    \lim_{t \to \infty} e(t) = \lim_{s \to 0} \left( \frac{s}{s} \cdot \frac{\frac{c_p}{m}s^2 + \frac{c_v}{m}s + \frac{c_p}{m}}{s^2 + \frac{c_v}{m}s + (1+k)\frac{c_p}{m}} \right) = \frac{1}{1+k}
\end{equation}
Therefore as $k \to \infty$, the steady state error goes to zero.
This error is also independent of parameters and so should not be affected by uncertainty.

However the damping ratio and frequency are strongly affected by these parameters.
\begin{equation}
    s = \sigma \pm j\omega = -\frac{c_v}{2m} \pm j \frac{1}{2} \sqrt{4(1+k)\frac{c_p}{m} - \left(\frac{c_v}{m}\right)^2}
\end{equation}
\begin{equation}
    \zeta = \frac{-\sigma }{ \sqrt{\sigma^2 + \omega^2}}
\end{equation}
To gain insight into how parametric uncertainty impacts the system response uncertainty must be propagated as follows:
\begin{equation}
    \delta \omega = \frac{\partial \omega}{\partial c_p} \delta c_p + \frac{\partial \omega}{\partial c_v} \delta c_v \quad \delta \zeta = \frac{\partial \zeta}{\partial c_p} \delta c_p + \frac{\partial \zeta}{\partial c_v} \delta c_v
\end{equation}
The derivatives can be found analytically and are shown plotted against $k$ at otherwise nominal values in figure \ref{fig:u_propagation}.

\begin{figure}[H]
    \centering
    \includegraphics[width=0.6\textwidth]{figures/u_propagation.png}
    \caption{}
    \label{fig:u_propagation}
\end{figure}

\begin{figure}[H]
    \centering
    \includegraphics[width=0.6\textwidth]{figures/rlocus_G.png}
    \caption{}
    \label{fig:rlocus_G}
\end{figure}


\begin{thebibliography}{9}

    %Endres, SC, Sandrock, C, Focke, WW (2018) “A simplicial homology algorithm for lipschitz optimisation”, Journal of Global Optimization.
    
      \bibitem{handout}
      J. V. Taylor and J. C. Massey
      \emph{GA3 Heat Exchanger Handout}
      University of Cambridge,
      2024.
    
      \bibitem{HeatTransfer}
      Holman J. P.
      \emph{Heat Transfer. 10th ed.}
      McGraw-Hill,
      2010.
    
      \bibitem{HE_design}
      Sadik Kakac, Hongtan Liu, Anchasa Pramuanjaroenkij,
      \emph{Heat Exchangers: Selection, Rating, and Thermal Design, Third Edition}
      CRC Press,
      2012.
    
      \bibitem{SHGO}
      Endres, SC, Sandrock, C, Focke, WW,
      \emph{A Simplicial Homology Algorithm for Lipschitz Optimisation},
      Journal of Global Optimization,
      2018.
    
\end{thebibliography}

\end{document}