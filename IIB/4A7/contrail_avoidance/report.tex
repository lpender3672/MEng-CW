
% here we go agane

\documentclass{article}

\usepackage[utf8]{inputenc}

\usepackage{amsmath, bm}
\usepackage{graphicx}
\usepackage{amssymb}
\usepackage{float}
\usepackage{caption}
\usepackage{subcaption}
\usepackage{hyperref}
\usepackage{tikz}
\usepackage{layout}
\usepackage{wrapfig}

\usepackage[margin=1in]{geometry}
\usepackage{listings}
\usepackage{xcolor}
\usepackage{color, colortbl}
\usepackage{textgreek}
\usepackage{mathrsfs}

\usepackage[moderate]{savetrees}

% 12 pt font
\renewcommand{\normalsize}{\fontsize{12pt}{\baselineskip}\selectfont}

\begin{document}

\title{4A7 Report \\ Climate Impact of Contrail Avoidance}
\author{5739G}
\date{December 2024}
\maketitle

\section{Introduction}
% evaluate the net climate impact of contrail avoidance

\section{Objectives}

\begin{itemize}
    \item Understand and quantify the climate impact of contrail avoidance using simple contrail and climate models.
    \item Investigate the impact of different aircraft fleets and fuels on the climate impact of contrail avoidance.
    \item Discuss limitations of simple models and compare to more complex models within the literature.
\end{itemize}

\subsection{Cases}

\begin{itemize}
    \item Baseline case, contrail avoidance with current aircraft fleet
    \item If the aircraft fleet is assumed to instead be fueled by liquified natural gas (LNG), which you can approximate as methane
    \item If the gas turbines powering the fleet have a higher thermal efficiency (by an amount of your choosing) 
\end{itemize}

\section{Simplified Model}

\subsection{Emissions}

The radiative forcing due to emissions of greenhouse gases due to additional fuel burn from contrail avoidance can be estimated using the following simplified method.

In literature common units for carbon coming into or going out of the atmosphere is billions of tons of carbon per year ($\text{GtCyr}^{-1}$)
denoted $E(t)$, at the time $t$ in years. The carbon content of the atmosphere in units of billions of metric tons of carbon is denoted $C(t)$.
The current year is denoted $t=0$.
Measurements of E(t) and C(t) show that only roughly half of carbon emissions stay in the atmosphere, the rest is absorbed by the ocean and biosphere \cite{co2_modelling}.
This empirical observation is captured by the \emph{Constant Airborne Fraction Model}:

\begin{equation}
    \frac{dC}{dt} = k \left( E(t) - \frac{C(t)-C_\text{pre}}{\tau}\right)
\end{equation}

Where $k \approxeq 0.5$, $\tau \approxeq 200 $ years and $C_\text{pre} \approxeq 600$ GtC.

Radiative forcing can be estimated using relations by G. Myhre \cite{rf_greenhouse}.

\begin{equation}
    \text{RF}_{\text{CO}_2} = 5.35 \ln \left( \frac{C}{C_0} \right), \quad \text{RF}_{\text{CH}_4} = 0.036 \left( \sqrt{M}-\sqrt{M_0} \right)
\end{equation}
For $N_2O$ the radiative forcing is given by:
\begin{eqnarray}
    \text{RF}_{\text{N}_2\text{O}} = 0.12(\sqrt{N} - \sqrt{N_0}) - (f(M_0,N) - f(M,N_0)) \\
    f(M, N) = 0.47 \ln\left[1 + 2.01 \times 10^{-5} (MN)^{0.75} + 5.31 \times 10^{-15} M (MN)^{1.52}\right]
\end{eqnarray}
Where contrentrations for M and N are in ppb.

\subsection{Contrails}

The radiative forcing due to contrails is extrapolated from the work of D.S. Lee et al. \cite{contrail_radiative_forcing}.

\subsection{Temperature response}

These radiative forcing values provide an instantaneous metric for climate impact of emissions, however
the temperature response is a cumulative effect of radiative forcing over time.
This can simply be modelled by two heat reservoirs, one for the atmosphere and one for the ocean.
The temperature of the atmosphere $T_1$ and the temperature of the ocean $T_2$ are then given by the following equations.

\begin{align}
    C_1 \frac{dT_1}{dt} &= \text{RF} - \lambda(T - T_0) - k_m(T_1 - T_2) \\
    C_2 \frac{dT_2}{dt} &= k_m(T_1 - T_2)
\end{align}

Where $\lambda$ is the climate sensitivity parameter, $T_0$ is the pre-industrial temperature, $k_m$ is the heat exchange coefficient between the atmosphere and the ocean, and $C_1$ and $C_2$ are the heat capacities of the atmosphere and ocean respectively.

AGWP can quantify cumulative radiative forcing.
AGTP can quantify the cumulative impact of increased temperatures such as changes in ecosystems and sea levels.
\begin{align}
    \text{AGWP} &= \int_0^T \text{RF}(t) dt \\
    \text{AGTP} &= \int_0^T \Delta T(t) dt
\end{align}

\subsection{Adaption for LNG fuel and higher thermal efficiency}

\subsection{Assumptions and Limitations}



\section{Complex Model}
% a list of all referenced models used in the report
% full description of model assumptions, limitations, and validation

% model of impact of contrails on climate

% require: fuel penalty associated with avoidance
%    meteorological conditions, aircraft type, altitude, and time of day
%    optimal routing and control strategies

% for additional cases may require different fuel consumption models
% but also the climate impact of sourcing methane from natural gas

\section{Results}
% tables and figures of results

\section{Discussion}
% discussion of results whatever they look like

\section{Conclusion}


\begin{thebibliography}{9}


    \bibitem{co2_modelling}
    R. H. Socolow and S. H. Lam
    \emph{Good enough tools for global warming policy making}
    Department of Mechanical and Aerospace Engineering, Princeton University,
    Princeton, NJ 08544, USA

    \bibitem{rf_greenhouse}
    G. Myhre et al.
    \emph{New estimates of radiative forcing due to well mixed greenhouse gases}
    Geophysical Research Letters, Vol. 25, No. 14, 1998

    \bibitem{contrail_radiative_forcing}
    D. S. Lee et al.
    \emph{The contribution of global aviation to anthropogenic climate forcing for 2000 to 2018}
    Atmospheric Environment, 2020

    \bibitem{lagentrainment}
    Green J E, Weeks D J and Brooman J W F,
    \emph{Prediction of turbulent boundary-layers and wakes in compressible flow by a lag-entrainment method.}
    ARC

    \bibitem{esdu}
    ESDU 99032
    \emph{VGK Method for Two-Dimensional Aerofoil Sections. Part 5. Design to Specified Upper Surface Distribution}
  
\end{thebibliography}

\end{document}