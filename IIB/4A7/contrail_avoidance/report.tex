
% here we go agane

\documentclass{article}

\usepackage[utf8]{inputenc}

\usepackage{amsmath, bm}
\usepackage{graphicx}
\usepackage{amssymb}
\usepackage{float}
\usepackage{caption}
\usepackage{subcaption}
\usepackage{hyperref}
\usepackage{tikz}
\usepackage{layout}
\usepackage{wrapfig}

\usepackage[margin=1in]{geometry}
\usepackage{listings}
\usepackage{xcolor}
\usepackage{color, colortbl}
\usepackage{textgreek}
\usepackage{mathrsfs}
\usepackage{savetrees}

% 12 pt font
\renewcommand{\normalsize}{\fontsize{12pt}{\baselineskip}\selectfont}

\begin{document}

\title{4A7 Report \\ Climate Impact of Contrail Avoidance}
\author{lwp26}
\date{December 2024}
\maketitle

\section{Introduction}
% evaluate the net climate impact of contrail avoidance

\subsection{Cases}

\begin{itemize}
    \item Baseline case, contrail avoidance with current aircraft fleet
    \item If the aircraft fleet is assumed to instead be fueled by liquified natural gas (LNG), which you can approximate as methane
    \item If the gas turbines powering the fleet have a higher thermal efficiency (by an amount of your choosing) 
\end{itemize}

\section{Models}
% a list of all referenced models used in the report
% full description of model assumptions, limitations, and validation

% model of impact of contrails on climate

% require: fuel penalty associated with avoidance
%    meteorological conditions, aircraft type, altitude, and time of day
%    optimal routing and control strategies

% for additional cases may require different fuel consumption models
% but also the climate impact of sourcing methane from natural gas

\section{Results}
% tables and figures of results

\section{Discussion}
% discussion of results whatever they look like

\section{Conclusion}


\begin{thebibliography}{9}

    \bibitem{handout}
    J. Jarret
    \emph{4A7 Transonic Wing Design Handout}
    University of Cambridge,
    2024.

    \bibitem{SA1_report}
    L. W. Pender
    \emph{SA1 Wing Analysis Final Report}
    University of Cambridge,
    2024.

    \bibitem{lagentrainment}
    Green J E, Weeks D J and Brooman J W F,
    \emph{Prediction of turbulent boundary-layers and wakes in compressible flow by a lag-entrainment method.}
    ARC

    \bibitem{esdu}
    ESDU 99032
    \emph{VGK Method for Two-Dimensional Aerofoil Sections. Part 5. Design to Specified Upper Surface Distribution}
  
\end{thebibliography}

\end{document}